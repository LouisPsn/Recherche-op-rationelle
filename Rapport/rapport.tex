%%%%%%%%%%%%%%%%%%%%%%%%%%%%%%%
%   Importation des paquets   %
%%%%%%%%%%%%%%%%%%%%%%%%%%%%%%%

% Utilisation de la classe "report" pour le document
\documentclass[11pt,a4paper,oneside]{article}

%Configuration de l'encodage et des polices du document
\usepackage[T1]{fontenc} 
\usepackage[utf8]{inputenc}
\usepackage{lmodern}

\renewcommand{\familydefault}{\sfdefault}

% Configuration du français
\usepackage[french]{babel}
\newcommand{\Numero}{\No}
\newcommand{\numero}{\no}
\setcounter{tocdepth}{2}
\renewcommand{\contentsname}{Sommaire} % si tableofcontents au début

%Configuration de la mise en forme du document
\usepackage[hmargin=3cm,vmargin=2.5cm,headheight=14pt]{geometry}
\usepackage{xspace}
\usepackage[table]{xcolor}
\usepackage{titlesec, blindtext, color}
\usepackage{lastpage}
\usepackage{fancyhdr}
\usepackage{lipsum}

% Gestion des tableaux
\usepackage{booktabs}
\usepackage{multicol}
\usepackage{multirow}

\usepackage{hyperref}
\hypersetup{
    pdftitle={Overleaf Example},
    }

% Gestion des figures et des images
\usepackage{graphicx}
\usepackage{subcaption}
\graphicspath{{./img}}

% Gestions des écritures mathématiques et des graphes
\usepackage{amsmath,amssymb,amsthm, amsfonts}
\usepackage{systeme}
\usepackage{stmaryrd}

\usepackage[french]{algorithm2e}

% Listings des codes
\usepackage[]{minted}
\usepackage{pythontex}
\usepackage{listings}

%%% N'oubliez pas les espaces devant les doubles ponctuations
% \NoAutoSpaceBeforeFDP
\def\subsectionautorefname~#1\null{%
  sous-section~#1\null
}

%%%%%%%%%%%%%%%%%%%%%%%%%%%%%%%%%%
%   Ajout de divers raccourcis   %
%%%%%%%%%%%%%%%%%%%%%%%%%%%%%%%%%%

\include{shortcuts.tex}

%%%%%%%%%%%%%%%%%%%%%%%%%%%%%%%%%%%%%%
%   Modifications de mise en forme   %
%%%%%%%%%%%%%%%%%%%%%%%%%%%%%%%%%%%%%%

% Redéfinition du titre de chapitre
\newcolumntype{M}[1]{>{\raggedright}m{#1}}
\addto\captionsfrench{\def\partname{}}
\definecolor{gray75}{gray}{0.75}
\makeatletter
\def\@makechapterhead#1{%
  {\parindent \z@ \Huge\bfseries
      \interlinepenalty\@M
      \thechapter\hsp\textcolor{gray75}{|}\hsp \Huge\bfseries #1 \hfill
      \vskip -15pt
      \HRule
      \vskip 15pt
  }}
\makeatother
\renewcommand{\thesubsection}{\alph{subsection})}

% Définition des traits séparateurs du document
\setlength{\parskip}{1ex plus 0.5ex minus 0.2ex}
\setlength{\columnseprule}{1pt}
\newcommand{\hsp}{\hspace{20pt}}
\newcommand{\HRule}{\rule{\linewidth}{0.5mm}}

% En-têtes et pieds de page
\pagestyle{fancy}
\renewcommand\headrulewidth{1pt}
\fancyhead[L]{Projet : Problème du bin-packing avec objets fragiles}
%\fancyhead[RE]{XXXX}
\fancyhead[R]{EI6IF127 Recherche Opérationnelle}
%\fancyhead[R]{\@author}
\renewcommand\footrulewidth{1pt}
\fancyfoot[C]{\textbf{\thepage/\pageref{LastPage}}}

\fancypagestyle{plain}{%
\fancyhf{}% clear all header and footer fields
\fancyfoot[C]{\textbf{\thepage/\pageref{LastPage}}}
\renewcommand{\headrulewidth}{0pt}%
\renewcommand{\footrulewidth}{0pt}%
}

%%%%%%%%%%%%%%%%%%%%%%%%%
%   Début du document   %
%%%%%%%%%%%%%%%%%%%%%%%%%

\begin{document}
% Insertion de la page de garde et de la table des matières
% Informations pour la page de garde
\title{Projet : Problème du bin-packing avec objets fragiles}
\def\authorone{\bsc{Laboirie} Alex}
\def\authortwo{\bsc{Pierson} Louis}

\def\teacher{\bsc{Larroze-Jardine} Sarah}
\date{\today}
\makeatletter

\begin{titlepage}
    \centering

    \includegraphics[width=0.6\textwidth]{logo.jpg}~\\[2.5cm]

    {\Large EI6IF127 Recherche Opérationnelle }\\[1.3cm]

    % Titre
    \HRule \\[0.4cm]

    { \huge \bfseries \@title \\[0.4cm]}

    \HRule \\[1cm]

    % Auteur et Enseignant
    \begin{minipage}{0.4\textwidth}
        \begin{flushleft} \large
            \emph{Auteurs~:} \authorone \\
            \phantom{\emph{Auteurs~:}} \authortwo \\
            \phantom{\emph{Auteurs~:}} \authorthree \\
            \phantom{\emph{Auteurs~:}} \authorfour \\
        \end{flushleft}
    \end{minipage}
    \begin{minipage}{0.5\textwidth}
        \begin{flushright} \large
            \emph{Chargé de TD~:} \teacher\\
            \phantom{\emph{Chargé de TD~:} \teacher}\\
        \end{flushright}
    \end{minipage}\\
    \vfill
%    \begin{minipage}{0.9\textwidth}%
%        \begin{abstract}
%            Ce document est mon modèle de rapport que je maintiens et fait évoluer en fonction de mes ù
%            besoins. Il n'a évidemment pas pour
%            objectif d'être universelle ou un modèle à suivre. Mais libre à quiconque de se servir de ce %modèle pour ses propres rapports.
%        \end{abstract}
%    \end{minipage}
%   \vfill
    % Bottom of the page

\vfill

    {\large \today}

\end{titlepage}

\cleardoublepage

% Insertion des exercices du rapport
\section{Première formulation}\label{sec:1_form}

\subsection*{Question 1.}

On peut borner le nombre de paquets dont on a besoin par $n$ car il y a $n$ objets.\vspace{12pt}


\subsection*{Question 2.}

On peut modéliser le problème sous la forme d'un programme linéaire comme ceci :\vspace{12pt}

\begin{equation}
    \begin{cases}
        min. z = \sum_{k = 1}^{n} y_k\\
        s.c. \hspace{6pt}\sum_{k = 1}^{n} x_{i, k} = 1 \hspace{12pt} \forall i \in \llbracket  1, n \rrbracket\\
        \hspace{25pt}\sum_{i = 1}^{n} x_{i, k}*w_i \leq f_i + (1 - annul_u)*M \hspace{12pt} \forall k \in \llbracket  1, n \rrbracket\\
        \hspace{25pt}x_{i, j} \leq y_k hspace{12pt} \forall i \in \llbracket  1, n \rrbracket, \forall k  \in \llbracket  1, n \rrbracket
    \end{cases}
\end{equation}


\subsection*{Question 3.}

\begin{lstlisting}[language=Python]
# Creation des variables x et y
x = [model1.add_var(name="x(" + str(i) + ")", lb=0, ub=1, 
    var_type=BINARY) for i in range(nb_objets*nb_boites)]
y = [model1.add_var(name="y(" + str(k) + ")", lb=0, ub=1,   
    var_type=BINARY) for k in range(nb_boites)]
annul = [model1.add_var(name="annul(" + str(k) + ")", lb=0, ub=1, 
    var_type=BINARY) for k in range(nb_objets)]

# Ajout de la fonction objectif au modele
model1.objective = minimize(xsum(y[k] for k in range (nb_boites)))


# Ajout des contraintes au modele
for i in range(nb_objets):
    [model1.add_constr(xsum([x[i + k*nb_objets] for k in 
        range(nb_boites)]) == 1)]

for k in range (nb_objets):
    [model1.add_constr(xsum(poids[i]*x[i + k*nb_objets] for 
        i in range(nb_boites)) <= fragilite[i] + (1 - annul[i])*M)]

for i in range (nb_objets):
    for k in range(nb_boites):
        [model1.add_constr(x[i + k*nb_boites] <= y[k])] 
\end{lstlisting}

Nous pouvons prendre $M = \sum_{i = 1}^{n} f_i$ comme majorant de toutes les fragilités.\vspace{12pt}

Dans la suite nous pourrons prendre $M = max(f_i)$, cette valeur ne respecte pas le modèle mais elle est facilement remplaçable à l'ordinateur par une valeur constante et donc qui respecte le modèle linéaire.\vspace{12pt}

\subsection*{Question 4.}

Il semble en effet intéressant de remplir les boites par ordre croissant vu que le nombre d'objets que l'on peut mettre dans les boites ne dépend pas de la boite en elle même mais des objets à l'intérieur.\vspace{12pt}

Pour réaliser ce remplissage des boites par ordre croissant il suffit d'ajouter la sous-condition :
$y_k \leq y_{k + 1}$ \hspace{12pt}$\forall k \in \llbracket  1, n-1 \rrbracket$

\subsection*{Question 5.}
\section{Seconde formulation}\label{sec:2_form}

\subsection*{Question 6.}

\begin{equation}
    \begin{cases}
        min. z = \sum_{i = 1}^{n} r[i]\\
        s.c. \hspace{6pt}\sum_{j = 1}^{n} z_{i,j} = 1 \hspace{12pt} \forall i \in \llbracket  1, n \rrbracket\\
        \hspace{25pt}\sum_{i = 1}^{n} z_{i, j}*w_i \leq f_j*r_j \hspace{12pt} \forall j \in \llbracket  1, n \rrbracket\\
        \hspace{25pt}z_{i, j} \leq r[j] \hspace{12pt} \forall i \in \llbracket  1, n \rrbracket, \forall j \in \llbracket  1, n \rrbracket
    \end{cases}
\end{equation}

La condition de minimisation représente ici un nombre minimal de paquet. Comme nous traitons ici avant des représentants d'objets ce sont eux qu'il faut minimiser.\vspace{12pt}

La sous-condition n°1 signifie que chaque objet doit avoir un seul et unique représentant.\vspace{12pt}

La sous-condition n°2 signifie que le poids total des objets ayant le même représentant doivent avoir un poids inférieur ou égal à la fragilité de leur représentant.\vspace{12pt}

Enfin la sous-condition n°3 signifie que si un objet est affecté à l'objet j alors l'objet j est un représentant.\vspace{12pt}


\subsection*{Question 7.}

On obtient donc cette implémentation en python :\vspace{4pt}

\begin{lstlisting}[language=Python]
    # Creation du modele vide
    model2 = Model(name = "BPFO_2", solver_name="CBC")  
    
    # Creation des variables z et y
    r = [model2.add_var(name="r(" + str(i) + ")", lb=0, ub=1, 
        var_type=BINARY) for i in range(nb_objets)]
    z = [model2.add_var(name="z(" + str(i) + ")", lb=0, ub=1, 
        var_type=BINARY) for i in range(nb_objets*nb_objets)]
    
    # Ajout de la fonction objectif au modele
    model1.objective = minimize(xsum(r[i] for i in range (nb_objets)))
    
    # Ajout des contraintes au modele
    for i in range(nb_objets):
        [model1.add_constr(xsum(z[i + j*nb_objets] for j in range(nb_objets)) == 1)]
    
    for j in range(nb_objets):
        [model1.add_constr(xsum(poids[j]*z[i + j*nb_objets] for i in range 
            (nb_objets)) <= fragilite[j]*r[j])]
    
    for j in range(nb_objets):
        for i in range(nb_objets):
            z[i + j*nb_objets] <= r[j]
\end{lstlisting}

Ici nous n'avons pas besoin d'introduire de variable $M$ car comme l'on connaît la fragilité minimale des objets du groupe d'objets grâce aux représentants.\vspace{12pt} 

\subsection*{Question 8.}

\subsection*{Question 9.}

\end{document}